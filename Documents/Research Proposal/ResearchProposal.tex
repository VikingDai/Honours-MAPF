\documentclass[a4paper,11pt]{article}

% set up sensible margins (same as for cssethesis)
\usepackage[paper=a4paper,left=30mm,right=30mm,top=25mm,bottom=25mm]{geometry}
\usepackage{natbib} % Use the natbib bibliography and citation package
\usepackage{setspace} % This is used in the title page
\usepackage{graphicx} % This is used to load the crest in the title page

% non-template packages
\usepackage{paralist}
\usepackage{multicol}
\usepackage{caption}
\usepackage{tabularx, booktabs}
\newcolumntype{Y}{>{\centering\arraybackslash}X}

\usepackage[font={small,it}]{caption}
\usepackage{hyperref}
\usepackage{xcolor}
\usepackage{lscape}
\hypersetup{
	colorlinks,
	linkcolor=teal,
	citecolor=teal,
	urlcolor=blue
}

%tikz stuff
\usepackage{tikz}
\usetikzlibrary{shapes, arrows, trees}
\tikzstyle{decision} = [diamond, draw, fill=green!20, text width=4.5em, text badly centered, node distance=3cm, inner sep=0pt]
\tikzstyle{block} = [rectangle, draw, fill=yellow!20, text width=3cm, text centered, rounded corners, minimum height=4em]
\tikzstyle{line} = [draw, -latex']
\tikzstyle{straight} = [draw]


\usepackage{array}
\newcolumntype{L}[1]{>{\raggedright\let\newline\\\arraybackslash\hspace{0pt}}m{#1}}
\newcolumntype{C}[1]{>{\centering\let\newline\\\arraybackslash\hspace{0pt}}m{#1}}
\newcolumntype{R}[1]{>{\raggedleft\let\newline\\\arraybackslash\hspace{0pt}}m{#1}}


%\hypersetup{
%	colorlinks,
%	linkcolor={red!50!black},
%	citecolor={blue!50!black},
%	urlcolor={blue!80!black}
%}

\begin{document}

% Set up a title page
\thispagestyle{empty} % no page number on very first page
% Use roman numerals for page numbers initially
\renewcommand{\thepage}{\roman{page}}

\begin{spacing}{1.5}
\begin{center}
{\Large \bfseries
School of Computer Science \\
Monash University}

\vspace*{30mm}

\includegraphics[width=5cm]{graphics/MonashCrest.pdf}

\vspace*{15mm}

{\large \bfseries
Research Proposal --- Comp Sci Honours, 2017
}

\vspace*{10mm}

{\LARGE \bfseries
Exploring improvements to Part-to-picker systems in Warehouses 
}

\vspace*{20mm}

{\large \bfseries
Phillip Wong 25150510

\vspace*{20mm}


%Supervisors: \parbox[t]{50mm}{Daniel Harabor}, \\Another person}
Supervisor: Daniel Harabor
}

\end{center}
\end{spacing}

\newpage

\tableofcontents

\newpage
% Now reset page number counter,and switch to arabic numerals for remaining
% page numbers 
\setcounter{page}{1}
\renewcommand{\thepage}{\arabic{page}}

	\begin{abstract} %100-200 words
	
	% Why
	\noindent The order picking process is the number one expense in the operating cost of warehouse systems. This project will look at \textit{part-to-picker}, a method of order picking where products are autonomously retrieved and delivered to the picking areas. Previous research has improved on multi-agent path finding (MAPF) algorithms but mostly overlooked the potential benefits gained by configuring the warehouse layout.
	% What
	Here, we will be exploring a number of adjustments and additions to Kiva systems which we expect to greatly determine what makes a good warehouse layout.
	% How
	These include: introducing an intermediate dropping zone, adding the capability for robots to maneuver under storage pods and optimizing order processing.
	% Where
	The results of this project will help identify how we should layout storage and picking stations in a warehouse. Additionally we will be looking at developing a MAPF method which uses a pre-computed path oracle.
	
\end{abstract}
\section{Introduction}
Order picking is a process in warehouse systems whereby products are retrieved from storage to satisfy incoming customer orders. This process has been identified by \cite{de2007design} as the most expensive process in operating a warehouse, estimated to take 55\% of the warehouse operating cost.

Here we look at a method of order picking known as part-to-picker systems. Part-to-picker systems contain multiple picking stations located around the warehouse. Products are brought to picking stations where workers will manually pick and process the product. One of the disadvantages of part-to-picker systems is that there will be some downtime at the picking stations while waiting for an order to be delivered. To solve this, these systems often use an automated storage and retrieval system (AS/RS). \cite{introduction2015autostore} is a recent part-to-picker system where products are organized in a grid of stacked bins. Robots move around the top of the grid, lifting bins and delivering them to a human picker. Benefits of the AutoStore system include high storage density and expansion capability. While not much literature is published about the specifics of AutoStore, we suspect the major downsides are: slow, expensive order retrieval as well as high infrastructure and maintenance costs.

In this project, we look at Kiva Systems (now known as Amazon Robotics). In Kiva systems, products are stored in shelves known as storage pods. Robots known as drive units are responsible for retrieving and delivering storage pods to picking stations before returning them to an appropriate place in the warehouse. A human worker is stationed at each picking station who picks the item off the pod before processing it (Fig \ref{kivaprocess}).

\begin{figure}[h!]
	\centering
	\includegraphics[width=0.6\textwidth ]{graphics/kivaprocess}
	\caption{A worker picking an order from a storage pod. The orange robot underneath is the drive unit. (\cite{kivayoutube2010quietlogistics})}
	\label{kivaprocess}
\end{figure}

%\noindent The process of order retrieval for a drive unit is as follows:
%\begin{compactenum}
%	\item Unit is told to retrieve a product
%	\item Unit moves to the storage pod containing the product and picks up the pod
%	\item Unit carries the pod to a picking station
%	\item Human worker picks the product from the pod and packs it
%	\item Unit returns the pod back to where it was picked up
%	\item Unit is told to retrieve a product
%\end{compactenum}

Kiva systems do not require a complex infrastructure to operate, a warehouse needs only a suitible number of storage pods, picking stations and drive units to operate. As long as the warehouse has space, more robots, pods or stations can be easily be added to the system to satisfy the incoming flow of customer orders. When a drive unit malfunctions it can be easily accessed and replaced. In summary, the main benefits of Kiva systems are their low initial and maintenance costs and their rapid deployment and flexibility (\cite{wurman2008coordinating}).

\newpage
\subsection{Research questions}
We aim to explore two areas of Kiva systems, the layout and MAPF. These can be summarized in the following questions:

\begin{enumerate}
	\item How will the adjustments and additions below affect our decision when it comes to configuring the warehouse layout?
	\begin{compactitem}
		\item Implementing an optimized order process
		\item Adding the capability for drive units to maneuver under storage pods
		\item Adding an intermediate zone where drive units may drop off storage pods
	\end{compactitem}

\item How much faster will the MAPF search run by pre-computing paths and storing them in a path oracle?

\end{enumerate}

\section{Background}
\label{background}
% Description of MAPF problem, including objective function

In Kiva systems we face a multi-agent pathfinding (MAPF) problem. MAPF aims to find a path for each agent to their goal while ensuring that no path conflicts with another. MAPF has usage in video games, robotics (\cite{bennewitz2002finding}), search and rescue (\cite{konolige2006centibots}) and of course warehouse applications. When analyzing the efficiency of a MAPF algorithm we generally aim to reduce the makespan of the system. Additionally, in Kiva systems we want to reduce the downtime of picking stations.

% How hard is it?
Finding an optimal solution in MAPF is an NP-hard problem (\cite{yu2013structure}) and mostly has found usage in systems containing a small number agents. This is not an option as Kiva systems deal with hundreds of agents, for example the Office Supply company, Staples uses 500 robots in their $30000m^{2}$ center (\cite{guizzo2008three}). Here we look at finding a bounded suboptimal solution and this has been explored in Kiva systems by (\cite{cohen2016bounded}).

% What are the main way people solve it?
To improve MAPF, generally methods are created to simplify the problem, \cite{cohen2016bounded} define user-provided highways to help guide agents towards a specific direction, greatly reducing the chance of path collisions. \cite{wilt2014spatially} identifies bottlenecks in the environment and assigns a controller which handles agents who want to pass through the bottleneck, simplifying behaviour in high collision zones. Another common technique is grouping agents into teams. \cite{ma2016optimal} splits agents into teams of 5 and presents a Conflict-Based Min-Cost-Flow algorithm which and shows that they can achieve a correct, complete and optimal solution.

% What are the main advantages and drawbacks of each approach?

%\cite{de2007design} provides a great overview of picking
Specific to the process of order picking, we can look at the method of order processing. Take an example where products of the same type are grouped together in a warehouse. If a large order of one product comes in, the agents will all try to find a path to this one area and create many collisions in the MAPF. We want the goal locations for our drive units to be spread evenly around the warehouse and order processing allows this by looking at two areas. Firstly, by smartly distributing products around the warehouse. If we place products of the same type across many pods row around the warehouse, a large order of one product will be no issue. Secondly, is sequencing of incoming orders. Instead of processing the large orders of one product, we have some flexibility to interlace this large order with other orders which we know we will need to process at some time, once again solving this issue. \cite{boysen2017parts} looks at both of these aspects in unison and found that with optimized order processing, only half the units are required to provide the supply given by a non-optimized system.

%it is possible to assist MAPF by manipulating incoming orders to suit the location of products. For example if a large order of one product came in, this could be interlaced with other orders, so agents do not need to all head to the same location. In a similar fashion, MAPF can be assisted by smart distribution of products across the warehouse. Take the same example where a large order of one product comes in, if the products were spread out around the warehouse this would not be an issue. This method is especially relevant in Kiva systems as storage pods can be easily moved around the warehouse by the drive units. \cite{boysen2017parts} looks at both of these aspects and found that with optimized order processing, only half the units are required to provide the supply given by a non-optimized system.

%\cite{wurman2008coordinating} provides an in depth overview of Kiva Systems, describing their benefits, usages and research areas.

%\cite{gu2010research} provides a comprehensive review of warehouse design and performance. It covers 5 major aspects, overall structure, sizing and dimensioning, department layout, equipment selection and operation strategy selection.

%\cite{de2007design} provides a survey on order picking

%\cite{strasser2015compressing} uses Compressed Path Databases.

%Unlike existing literature, in this project we aim looking at a number of other factors which are likely to simplify the pathfinding problem.

%Windowed Hierarchical Cooperative A∗. Cooperative A*. Conflict-Oriented Windowed Hierarchical Cooperative A∗. Compressed Path Databases.

\section{Methodology}
\label{method}

\subsection{Warehouse layout}
\label{warehouselayout}
In previous studies we usually see the picking station positioned on one side of the warehouse and pods are laid out in rows (Fig. \ref{kivalayout1}). Warehouse layout looks at configuring the location of both storage pods and picking stations. We will determine a good warehouse layout to be one which minimizes makespan as well as downtime of picking stations. Firstly, we will investigate by looking at a user-defined set of warehouse layouts. A possibility later in the project is the use of genetic algorithms to find a good warehouse layout. 

\begin{figure}[h]
	\centering
	\includegraphics[width=0.9\textwidth]{graphics/kivasystemlayout}
	\caption{A Small Region of a Kiva Layout (\cite{wurman2008coordinating}). Picking stations located on the left and storage pods laid out in rows.}
	\label{kivalayout1}
\end{figure}

\noindent Below we describe a number of adjustments and additions which are expected to play a large role in determining how we create a good warehouse layout.

\subsubsection{Intermediate zone}
\label{intermediatezone}
In Section \ref{background}, we saw that \cite{wilt2014spatially} identified bottlenecks in the environment and assigned controllers to simplify behaviour for any agents who wanted to travel through the bottleneck. Inspired by this, we plan to split the warehouse into two halves and introduce an intermediate zone (See Fig \ref{kivalayout2}). Now, drive units located in the far zone deliver pods to the intermediate zone instead of a pickup station. Drive units located in the delivery zone will be additionally responsible for retrieving pods in the intermediate zone and bringing them to pickup stations. We suspect this to provide benefits in traditional warehouse layouts where we have picking stations on one side and storage pods laid in rows.

\begin{figure}[h]
	\centering
	\includegraphics[width=0.9\textwidth]{graphics/kivasystemlayout_adjusted}
	\caption{Intermediate zone in red, delivery zone in blue and far zone in green}
	\label{kivalayout2}
\end{figure}

\subsubsection{Order processing}
\label{orderprocessing}
This process is described in detail in Section \ref{background}. Reiterating the main points of order processing we look at distribution of products around the warehouse to reduce path collisions and smart sequencing of incoming orders. This is important to consider as we expect that an optimized order processing sequence will decrease the effects of having a poor warehouse layout. Here we may take inspiration from Robin-hood hashing as well as the previous work from \cite{boysen2017parts}.

\subsubsection{Capability for movement underneath storage pods}
\label{beneathpods}
One of the specifications for creating a drive unit is the ability to maneuver underneath a storage pods to pick it up. This means that by slightly adjusting the dimensions of storage pods it is possible to allow drive units to maneuver underneath pods. Until a unit starts carrying a storage pod, it will not see storage pods as any obstacles in the environment. Essentially this simplifies the MAPF problem as we have decreased the number of obstacles and will allow for more  paths during retrieval. We expect this to reduce both the makespan of the system and downtime of picking stations. Moreover with this in place we suspect the warehouse layout to be less affected by the positioning of storage pods as they are not obstacles for most of the time.

\subsection{MAPF and the use of a path oracle}
\label{pathoracle}
Decentralised approaches are usually used in MAPF algorithms due to their lower CPU and memory requirements (\cite{wang2009bridging}) compared to a centralised approach. Decentralised MAPF involves agents independently searching for a path to their goal. Here we aim to introduce a path oracle which will pre-compute paths, removing the cost of performing a path searching at runtime and hence significantly improving performance. We expect to base this off the work by \cite{strasser2015compressing}, utilizing a Compressed Path Database.

Path collisions in MAPF is a type of spatio-temporal conflict. Here, we will explore two options to resolve them. When a collision occurs, we choose one agent based on a user-defined utility function and this agent will continue along their path. We have two methods of resolving these collisions, firstly we can make the other agent wait until the chosen agent moves out of the way. Alternatively we can use a spatio-temporal reservation table which looks forward to find a path without conflicts. This method will be based off the work by \cite{wilt2014spatially}.


\subsection{Timetable}

There are a large number of individual tasks involved in this project which have been spread across the two semesters. I have identified two major tasks which are: implementing an optimized order processing method and implementing path oracle. These have been allocated multiple weeks in semester 2 and the path oracle has been prioritized over optimized order processing as we will have implemented robin-hood hashing earlier in week 11 of semester 1. Finally, some time has been allocated to presentations or deliverables related to this project and two weeks have been allocated to focus on university examinations.

\begin{center}
{\footnotesize
\begin{tabular}{ c p{12cm} }
\multicolumn{2}{l}{\textbf{Semester 1}} \\
\hline \multicolumn{1}{c}{Week(s)} & \multicolumn{1}{c}{Plan} \\
\hline 7  & Create warehouse simulation with simple A* pathfinding. \\
\hline 8-9  & Add multiple agents to the simulation using with Cooperative A* and moving between pickup stations. Agents will wait during collisions (\ref{pathoracle}). \\
\hline 10 & Implement an order sequencer assigning generating an inflow of products to be retrieved. Implement agents using a reservation table to resolve collisions (\ref{pathoracle}). \\
\hline 11 & Implement robin-hood hashing (\ref{orderprocessing}) \\
\hline 12-14 & Focus on interim presentation, literature review and examinations \\
\hline Holidays & Look at adding an intermediate dropping zone (\ref{intermediatezone}). Implement ability to move under pods (\ref{beneathpods}). Begin implementing a path oracle (\ref{pathoracle}). \\
\hline
\end{tabular}
}

{\footnotesize
\vspace{0.5cm}
\begin{tabular}{ c p{12cm} }
\multicolumn{2}{l}{\textbf{Semester 2}} \\
\hline \multicolumn{1}{c}{Week(s)} & \multicolumn{1}{c}{Plan} \\
\hline 1-3 & Continue implementation of a path oracle (\ref{pathoracle}) \\
\hline 4-7 & Implement an optimized order processing method (\ref{orderprocessing}) based on \cite{boysen2017parts} \\
\hline 8-11 & Run simulation looking at combination of modifications and analyze results. Use any extra time to explore the use of genetic algorithms to generate warehouse layouts (\ref{warehouselayout}). \\
\hline 12 & Finish first draft of Final Thesis \\
\hline 13-15 & Focus on final presentation, final thesis and examinations \\
\hline
\end{tabular}
}
\end{center}

\section{Expected Outcomes}
Overall we aim to contribute one major insight and one deliverable. In Section \ref{method}, we described methods to add: an optimized order process, the capability for drive units to path under storage pods and the addition of an intermediate zone. What we aim to answer is how these modifications may affect our decision when it comes to designing a good warehouse layout. To showcase our results we hope to produce two main graphs comparing the makespan of the system and the idle time of picking stations across our user-defined warehouse layouts. Accompanying this would be any interesting finding we see from mixing and matching the adjustments and additions described in Section \ref{method}. 

For our deliverable, we hope to produce an improved MAPF solution utilizing a pre-computed path oracle and showcase its usage in a computer simulated Kiva system. We will compare it to other MAPF techniques, looking at the makespan of the system and idle time of picking stations.

\bibliographystyle{dcu}
\bibliography{bibliography}

\end{document}
