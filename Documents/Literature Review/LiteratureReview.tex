\documentclass[a4paper,11pt]{article}

% set up sensible margins (same as for cssethesis)
\usepackage[paper=a4paper,left=30mm,right=30mm,top=25mm,bottom=25mm]{geometry}
\usepackage{natbib} % Use the natbib bibliography and citation package
\usepackage{setspace} % This is used in the title page
\usepackage{graphicx} % This is used to load the crest in the title page

% non-template packages
\usepackage{paralist}
\usepackage{multicol}
\usepackage{caption}
\usepackage{tabularx, booktabs}
\newcolumntype{Y}{>{\centering\arraybackslash}X}

\usepackage[font={small,it}]{caption}
\usepackage{hyperref}
\usepackage{xcolor}
\usepackage{lscape}
\hypersetup{
	colorlinks,
	linkcolor=teal,
	citecolor=teal,
	urlcolor=blue
}

%tikz stuff
\usepackage{tikz}
\usetikzlibrary{shapes, arrows, trees}
\tikzstyle{decision} = [diamond, draw, fill=green!20, text width=4.5em, text badly centered, node distance=3cm, inner sep=0pt]
\tikzstyle{block} = [rectangle, draw, fill=yellow!20, text width=3cm, text centered, rounded corners, minimum height=4em]
\tikzstyle{line} = [draw, -latex']
\tikzstyle{straight} = [draw]


\usepackage{array}
\newcolumntype{L}[1]{>{\raggedright\let\newline\\\arraybackslash\hspace{0pt}}m{#1}}
\newcolumntype{C}[1]{>{\centering\let\newline\\\arraybackslash\hspace{0pt}}m{#1}}
\newcolumntype{R}[1]{>{\raggedleft\let\newline\\\arraybackslash\hspace{0pt}}m{#1}}

\usepackage{float}

%\hypersetup{
%	colorlinks,
%	linkcolor={red!50!black},
%	citecolor={blue!50!black},
%	urlcolor={blue!80!black}
%}

\begin{document}

% Set up a title page
\thispagestyle{empty} % no page number on very first page
% Use roman numerals for page numbers initially
\renewcommand{\thepage}{\roman{page}}

\begin{spacing}{1.5}
\begin{center}
{\Large \bfseries
School of Computer Science (BICA) \\
Monash University}


\vspace*{30mm}

\includegraphics[width=5cm]{graphics/MonashCrest.pdf}

\vspace*{15mm}

{\large \bfseries
Literature Review, 2017
}

\vspace*{10mm}

{\LARGE \bfseries
Improving cooperative pathfinding using a path oracle
}

\vspace*{20mm}

{\large \bfseries
Phillip Wong

\vspace*{20mm}


Supervisors: \parbox[t]{50mm}{Daniel Harabor,\\Pierre Le Bodic}
}

\end{center}
\end{spacing}

\newpage

\tableofcontents

\newpage
% Now reset page number counter,and switch to arabic numerals for remaining
% page numbers 
\setcounter{page}{1}
\renewcommand{\thepage}{\arabic{page}}

	\begin{abstract} %100-200 words
	
	% Why
	\noindent The order picking process is the number one expense in the operating cost of warehouse systems. This project will look at \textit{part-to-picker}, a method of order picking where orders are retrieved and delivered to a number of picking areas located around the warehouse. Previous research has improved on multi-agent path finding (MAPF) algorithms but mostly overlooked the potential benefits gained by configuring the warehouse layout.
	% What
	Here, we will be exploring Kiva systems a part-to-picker system which uses autonomous vehicles and mobile storage. Our focus is to explore a number of adjustments and additions which we expect will greatly affect how we design warehouse layouts.
	% How
	These include: introducing an intermediate dropping zone, optimizing order processing and adding the capability for robots to maneuver under storage pods.
	% Where
	The results of this project will help identify how we should position storage and picking stations in a warehouse. Additionally, we will be looking at developing a MAPF method which uses a pre-computed path oracle.
	
	\noindent \\ \textbf{Keywords} \\
	Cooperative Multi-agent pathfinding, Kiva systems
	
\end{abstract}
\section{Introduction}
Order picking is a process in warehouse systems whereby products are retrieved from storage to satisfy incoming customer orders. This process has been identified by \cite{de2007design} as the most expensive process in operating a warehouse, estimated to take 55\% of the warehouse operating cost.

Here we look at a method of order picking known as part-to-picker systems. Part-to-picker systems contain multiple picking stations located around the warehouse. Products are brought to picking stations where workers will manually pick and process the product. One of the disadvantages of part-to-picker systems is that there will be some downtime at the picking stations while waiting for an order to be delivered. To solve this, these systems often use an automated storage and retrieval system (AS/RS). \cite{introduction2015autostore} is a recent part-to-picker system where products are organized in a grid of stacked bins. Robots move around the top of the grid, lifting bins and delivering them to picking areas. Benefits of the AutoStore system include high storage density and expansion capability. While not much literature is published about the specifics of AutoStore, we suspect the major downsides are: slow, expensive order retrieval as well as high infrastructure and maintenance costs.

In this project, we look at Kiva Systems (now known as Amazon Robotics). In Kiva systems, products are stored in mobile shelves known as storage pods. Robots known as drive units are responsible for retrieving and delivering storage pods to picking stations. A human worker is stationed at each picking station who picks the item off the pod before processing it (Fig \ref{kivaprocess}). Once the pod has been processed, the drive unit will return the pod to an appropriate location in the warehouse.


%\noindent The process of order retrieval for a drive unit is as follows:
%\begin{compactenum}
%	\item Unit is told to retrieve a product
%	\item Unit moves to the storage pod containing the product and picks up the pod
%	\item Unit carries the pod to a picking station
%	\item Human worker picks the product from the pod and packs it
%	\item Unit returns the pod back to where it was picked up
%	\item Unit is told to retrieve a product
%\end{compactenum}

Kiva systems do not require a complex infrastructure to operate, a warehouse needs only a suitible number of storage pods, picking stations and drive units to operate. As long as the warehouse has space, more robots, pods or stations can be easily be added to the system to satisfy the incoming flow of customer orders. When a drive unit malfunctions it can be easily accessed and replaced. In summary, the main benefits of Kiva systems are their low initial and maintenance costs and their rapid deployment and flexibility (\cite{wurman2008coordinating}).



\newpage
\subsection{Research questions}
We aim to explore two areas of Kiva systems, the layout and MAPF. These can be summarized in the following questions:

\begin{enumerate}
	\item How will the adjustments and additions below affect our decision when it comes to configuring the warehouse layout?
	\begin{compactitem}
		\item Adding an intermediate zone where drive units may drop off storage pods
		\item Adding the capability for drive units to maneuver under storage pods
		\item Implementing an optimized order process

	\end{compactitem}

\item How much faster will the MAPF search run by pre-computing paths and storing them in a path oracle?
\end{enumerate}

\section{Background}
\label{background}
% Description of MAPF problem, including objective function

In Kiva systems, we face a multi-agent pathfinding (MAPF) problem. MAPF aims to find a path for each agent to their goal while ensuring that no path conflicts with another. MAPF has usage in video games, robotics (\cite{bennewitz2002finding}), search and rescue (\cite{konolige2006centibots}) and warehouse applications. When analyzing the efficiency of a MAPF algorithm we generally aim to reduce the makespan of the system. Additionally, in Kiva systems we want to reduce the downtime of picking stations.

% How hard is it?
Finding an optimal solution in MAPF is an NP-hard problem (\cite{yu2013structure}) and mostly has found usage in systems containing a small number agents. This is not an option as Kiva systems deal with hundreds of agents, for example the Office Supply company, Staples uses 500 robots in their $30000m^{2}$ center (\cite{guizzo2008three}). Here we look at finding a bounded suboptimal solution and this has been explored in Kiva systems by (\cite{cohen2016bounded}).

% What are the main way people solve it?
To improve MAPF, generally methods are created to simplify the problem, \cite{cohen2016bounded} define user-provided highways to help guide agents towards a specific direction, greatly reducing the chance of path collisions. \cite{wilt2014spatially} identifies bottlenecks in the environment and assigns a controller which handles agents who want to pass through the bottleneck, simplifying agent behaviour in high collision zones. Another common technique is grouping agents into teams. \cite{ma2016optimal} splits agents into teams of 5 and presents a Conflict-Based Min-Cost-Flow algorithm which and shows that they can achieve a correct, complete and optimal solution.

% What are the main advantages and drawbacks of each approach?

%\cite{de2007design} provides a great overview of picking
Specific to the process of order picking, we will look at the method of order processing. Take an example where products of the same type are grouped together in a warehouse. If a large order of one product comes in, the agents will all try to find a path to this one area and create many collisions in the MAPF. We want the goal locations for our drive units to be spread evenly around the warehouse and order processing allows this by looking at two areas. Firstly, by evenly distributing products around the warehouse. If we place products of the same type across many different row around the warehouse, a large order of one product will be no issue. Secondly, is sequencing of incoming orders. Instead of processing the large orders of one product sequentially, we have some flexibility to interlace this large order with other orders which we know we will need to process. Essentially, we can move the mobile storage pods as well adjust the incoming ordering sequence to benefit the MAPF. \cite{boysen2017parts} looks at both these aspects in unison and found that with optimized order processing, only half the units are required to provide the supply given by a non-optimized system.

\subsection{Objective function}

\cite{yu2015optimal} looked at: 
\begin{compactenum}
	\item Total Arrival time: 
	\item Makespan: The time between 
	\item Total Distance: 
	\item Maximum Distance: 
\end{compactenum}

\begin{enumerate}
	\item Test
\end{enumerate}



\subsubsection{}

\subsection{Improving A*}

\subsubsection{Compressed Path Databases}

\subsubsection{Jump Point Search}

\cite{renukamurthy2016improving}

\subsection{Linear programming}
Linear programming is commonly used when dealing with NP-hard problems. In pathfinding it excels at finding an optimal path. \textbf{In this technique, we look at an optimization function and constraints. A solver will work to find the optimal solution to the problem.}

The book by \cite{ahuja1993network}, introduces many algorithms and applications, most related to this research is Section 4 which overviews shortest path searches. \cite{planes2009path} applies integer programming to search through a metabolic pathways. Here they find a series of biochemical reactions which a living organism will transform an compound to the target compound.

MAPF

\cite{schouwenaars2001mixed} Multi-vehicle path planning. Mixed integer programming.

\cite{richards2002aircraft} Aircraft collision avoidance. Mixed integer linear programming.

\cite{surynek2016efficient} Satisfiability problem (SAT)

\cite{yu2013planning} Multi-robot path planning. Multiflow based integer linear programming (ILP).

Branch and price is a method for solving integer programming problems





\subsection{Comparing MAPF algorithms}

{\setlength{\parindent}{0cm}

\textbf{Completeness:} if a path to the goal exists, it is always found

\textbf{Solution quality:}
	\begin{compactitem}
		\item Optimal: finds the shortest path to the goal
		\item Bounded suboptimal: finds a path to the goal where the difference between the found path length and the optimal path length is always less than some bound
		\item Suboptimal: finds a path to the goal with no guarantee of path length
	\end{compactitem}

\textbf{Anonymous:} which agent goes to which goal is interchangeable

\textbf{Time complexity:}
	\begin{compactitem}
		\item Non-Polynomial (NP) / Intractable
		\item Polynomial / Tractable
	\end{compactitem}

\textbf{Dynamic?:} If the environment is changed does anything need to be recomputed i.e. did the algorithm require preprocessing

\textbf{Anytime:} The search can spend more computation to improve the solution quality or number of solutions. It can stop earlier at any time and return a path.

\textbf{Centralized / Coupled:} \textit{A centralised approach (Barraquand \& Latombe 1991; LaValle \& Hutchin- son 1998) has a single global decision maker for all agents, is theoretically optimal but, as discussed before, it does not scale up to many agents due to a prohibitive complexity. A decoupled (decentralized) approach decomposes the problem into several subproblems. The latter approach is faster but yields suboptimal solutions and loses the completeness} \cite{wang2008fast}

\textbf{Online / Offline} \textit{An online algorithm is one that can only process its input piece-by-piece in a serial fashion, in the order that the input is fed to the algorithm, without having the entire input available from the start. An offline algorithm is given the whole problem data from the beginning and is required to output an answer which solves the problem at hand.}
} % End no indent
\begin{figure}[H]
\centering
\small
\begin{tabular}{ l c c c c c }
	\textbf{Search} & \textbf{Centralised}  & \textbf{Complete} & \textbf{Tractable} & \textbf{SQ} & \textbf{Anon} \\
	\hline
	WHCA* 				& N & N & N & O  & No \\
	FAR  				& N & Y & Y & BO & No \\
	MAPP 				& N & N & O & BO &No \\
	CBS 				& N & N & O & O  & \\
	BCP					& N & & & & \\
	Centralised A* 		& Y & & & & \\	
\end{tabular}

\caption{Comparison of MAPF algorithms}
\end{figure}


%\cite{wurman2008coordinating} provides an in depth overview of Kiva Systems, describing their benefits, usages and research areas.

%\cite{gu2010research} provides a comprehensive review of warehouse design and performance. It covers 5 major aspects, overall structure, sizing and dimensioning, department layout, equipment selection and operation strategy selection.

%\cite{de2007design} provides a survey on order picking

%\cite{strasser2015compressing} uses Compressed Path Databases.

%Unlike existing literature, in this project we aim looking at a number of other factors which are likely to simplify the pathfinding problem.

%Windowed Hierarchical Cooperative A∗. Cooperative A*. Conflict-Oriented Windowed Hierarchical Cooperative A∗. Compressed Path Databases.

\subsection{Cooperative Multi-agent pathfinding}

When looking at multi-agent pathfinding, we will be first considering whether it is centralized. A centralized search usually indicates

\subsubsection{CA*, HCA* and WHCA*}
\cite{silver2005cooperative}
\subsubsection*{Cooperative A*}

\textit{The individual searches are performed in three dimensional space-time, and take account of the planned routes of other agents. A wait move is included in the agent’s action set, to enable it to remain stationary. After each agent’s route is calculated, the states along the route are marked into a reservation table. Entries in the reservation table are considered impassable and are avoided during searches by subsequent agents.
}
\subsubsection*{Hierarchical Cooperative A*}
\textit{Hierarchical Cooperative A* (HCA*) uses a simple hierarchy containing a single domain abstraction, which ignores both the time dimension and the reservation table. In other words, the abstraction is a simple 2-dimensional map with all agents removed. Abstract distances can thus be viewed as perfect estimates of the distance to the destination, ig- noring any potential interactions with other agents. This is clearly an admissible and consistent heuristic.}


\subsubsection*{Windowed Hierarchical Cooperative A*}
\textit{The cooperative search is limited to a fixed depth specified by the current window. Each agent searches for a partial route to its destination, and then begins following the route. At regular intervals (e.g. when an agent is half-way through its partial route) the window is shifted forwards and a new partial route is computed}


\subsubsection{FAR}
\cite{wang2008fast}

\textit{When building a search graph from a grid map, FAR implements a flow restriction idea inspired by road networks. The movement along a given row (or column) is restricted to only one direction, avoiding head-to-head collisions. The movement direction alternates from one row (or column) to the next. Additional rules ensure that two locations reachable from each other on the original map remain connected (in both directions) in the graph. After building the search graph, an A* search is independently run for each mobile unit. Dur- ing plan execution, deadlocks are detected as cycles of units that wait for each other to move. A heuristic procedure for deadlock breaking attempts to repair plans locally, instead of running a larger scale, more expensive replanning step.}

\subsection{Slidable}
\subsubsection*{MAPP}
\cite{wang2011mapp}

\textit{For each problem instance, MAPP systematically identifies a set of units, which can contain all units in the instance, that are guaranteed to be solved within low-polynomial time} \\

Based on sliding tile puzzle. Always tries to keep a blank on an alternative path.

\subsubsection*{TASS}
\textit{tree-based agent swapping strategy}

\section{Optimality}

\subsubsection{ICTS}

\subsubsection{Conflict-Based Search}
Conflict-Based Search (CBS) is a optimal search algorithm.

When a collision occurs a constraint 
Constraints describe the agent, time and location.

The algorithm uses a constraint tree to describe these constraints and searches through this tree.


\subsection{TAPF}

\textit{Combined target assignment and pathfinding (TAPF) couples the target-assignment and the pathfinding problems and defines one common objective for both of them. Agents are partitioned into teams.
Each team is given the same number of unique targets as there are agents in the team. The task of TAPF is to assign the targets to the agents and plan collision-free paths for the agents from their current vertices to their targets in a way such that each agent moves to exactly one target given to its team, all targets are visited and the makespan is minimized}

\subsection{CBM}
Conflict-based Min-cost flow

\textit{The anonymous variant of MAPF (also called goal- invariant MAPF). Agents can get assigned any target and are thus exchangeable. It can be solved optimally in polynomial time with flow-based MAPF methods}

\subsection{PERR}
Package-Exchange Robot-Routing (PERR)

\subsection{Highways}
\cite{cohen2016bounded}

\subsection{Operator Decomposition}
\textit{Standley proposes the mechanism of operator decomposition (OD) [30]. Instead of considering that the set of agents decides its joint action (operator), the agents decide their elementary action in turn, one after each other. With some care to special cases, A* associated with OD and a perfect heuristic is admissible and complete [30]}

\subsection{Independance Detection}
\textit{Each agent plans its optimal path to the goal with A*. Then conflicts are detected, and resolved if possible. When a conflict between two agents is not solvable, the two agents are gathered into a group, and an optimal solution is found for this group. Conflict between groups are detected and so forth. When few conflicts occur this method avoids the exponential size of the set of actions, but in the worst case, this method is inferior to any centralized method.}


\section{Other factors}
\cite{honig2016multi} factored in kinematic constraints.

\section{Warehouse}

\subsection{Overview papers}
\cite{gu2007research} (Research on warehouse operation: A comprehensive review)

\cite{gu2010research} (Research on warehouse design and performance evaluation: A comprehensive review)

\subsection{Summary}


\subsection{Further research}


\bibliographystyle{dcu}
\bibliography{bibliography}

\end{document}
